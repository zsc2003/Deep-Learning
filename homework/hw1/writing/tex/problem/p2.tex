\begin{homeworkProblem}

2. [3pts] Prove the shift invariant property of convolution:
$$\mathcal{G}(f(\cdot-a))(x)=\mathcal{G}(f(\cdot))(x-a),$$
where $\mathcal{G}$ is the convolutional operator $\mathcal{G}(f(\cdot))=\displaystyle\int_{-\infty}^{+\infty} f(\tau)h(x-\tau)\dtau$

\textcolor{blue}{Solution}

From the definition of convolution, we can get that
\begin{align*}
\mathcal{G}(f(\cdot - a))(x) &= \int_{-\infty}^{+\infty} f(\tau-a) h(x-\tau) \dtau \\
\text{(Let $t=\tau-a$)}\quad &= \int_{-\infty}^{+\infty} f(t) h(x-(t+a)) \dt \\
&= \int_{-\infty}^{+\infty} f(t) h((x-a)-t) \dt \\
&= \mathcal{G}(f(\cdot))(x - a)
\end{align*}

So above all, we have proved that the shift invariant property of convolution
$$\mathcal{G}(f(\cdot-a))(x)=\mathcal{G}(f(\cdot))(x-a)$$

\end{homeworkProblem}

\newpage